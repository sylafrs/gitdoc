\chapter{Notions avancées}
\section{Empêcher l'indexation avec .gitignore}

.gitignore est un fichier permettant d'empêcher l'indexation de fichiers (git add).\\
Ils fonctionnent sur le principe suivant : Chaque ligne du fichier contient une expression régulière. Si un fichier à un chemin qui correspond à l'une des expressions régulières d'un .gitignore (chemins relatifs), alors le fichier ne sera pas indexé (s'il ne l'est pas déjà).

Pour retirer un fichier du cache, il suffit d'utiliser git rm :
\begin{verbatim}
# Supprime DU CACHE tous les fichiers du dossier.
$ git rm -rf --cached .

# Ajoute dans le cache tous les fichiers du dossier.
$ git add .

# Les deux commandes mises côte à côte permet de sortir
# du cache tout ce qui est dans le .gitignore.
\end{verbatim}

Sous Tortoise git :\\

Clic-droit sur un fichier $\rightarrow$ Tortoise Git $\rightarrow$ Delete (keep local)

\section[Optimisez votre dépôt avec le Garbage Collector]{Optimisez votre dépôt avec le GC}

git gc et git prune

\section{Comparer deux versions}

git diff et git log