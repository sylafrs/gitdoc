% Preface
\section*{Preface}
Voici un document crée à partir de mes connaissances sur git et de quelques sources issues d'Internet. Je ne vais PAS détailler les commandes que j'utilise souvent ni même les fenêtres qui apparaîtrons dans Tortoise Git, une interface que nous allons utiliser.\\

Si vous voulez maîtriser git et connaître toutes ses commandes de fond en comble, vous devrez utiliser \href{https://www.kernel.org/pub/software/scm/git/docs/}{le manuel} ou bien internet.\\

La plupart des commandes git ont l'option --help :
\begin{verbatim}
$ git <commande> --help 
$ git help <commande>
$ man git-<commande>
\end{verbatim}

Si vous souhaitez connaître les options d'une fenêtre en particulier, le mieux reste de trouver quelle commande la fenêtre appellera et de regarder ses options sur internet.\\

Ce tutoriel est certes technique mais part de zéro : n'importe qui devrait pouvoir le lire. Si un passage n'est pas clair, je reste disponible à l'adresse sylvain.lafon.91@gmail.com : je vous expliquerai ledit passage et modifierai le document afin de retirer les ambiguïtés et autres passages obscurs.\\

Le tutoriel officiel de git se trouve dans le manuel :
\begin{itemize}
\item Première partie : \href{https://www.kernel.org/pub/software/scm/git/docs/gittutorial.html}{gittutorial (7)}
\item Seconde partie : \href{https://www.kernel.org/pub/software/scm/git/docs/gittutorial-2.html}{gitutorial-2 (7)}
\item \href{https://www.kernel.org/pub/software/scm/git/docs/user-manual.html}{Manuel complet}
\item \href{https://www.kernel.org/pub/software/scm/git/docs/everyday.html}{Commandes de base}\\
\end{itemize}

Ce livre peut également vous être très utile : \href{https://git-scm.com/book/fr/v2}{Pro Git} aussi disponible dans sa \href{https://progit2.s3.amazonaws.com/fr/2016-03-05-4c838/progit-fr.1062.pdf}{version PDF}
