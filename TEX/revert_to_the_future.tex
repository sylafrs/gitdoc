\newpage
\section{Retournons vers le futur}
\subsection{Reprendre la version d'un ou plusieurs fichiers précis}

Car oui, il est possible \textbf{sans tricher\footnote{Tricher : Se prendre la tête en se déplaçant dans les commits et des copiés collés}} de faire revenir juste un fichier en arrière, et ce, grâce à checkout !

\begin{verbatim}
# Remet fichier tel qu'il est dans <id> (ex : master, SHA-1, HEAD, tag)
$ git checkout <id> /chemin/fichier

# Remet fichier tel qu'il est dans le commit dans lequel HEAD est.
$ git checkout -- /chemin/fichier
\end{verbatim}

-- permet de séparer les $<$id$>$ des fichiers. Dans le cas où une branche s'appelle fichier.txt (oui, je sais), cela permet de retirer l'ambiguïté\\

Avec Tortoise git, on appelle ça : faire un Revert.

\begin{itemize}
\item Clic-droit sur un fichier de Working dir changes : annule les modifications courantes d'un fichier
\item Clic-droit sur un fichier d'un commit : fait revenir un fichier tel qu'il était dans le commit en question.
\end{itemize}

\subsection{Annuler un ou plusieurs commits}
Une autre commande peut-être utilisée si l'on veut prendre tous les fichiers à la fois : revert. Tortoise Git permet de le faire en faisant un clic doit sur le(s) commit(s) en question.

\begin{verbatim}
# Annule les changements du commit <id> et crée un commit.
$ git revert <id>

# Annule les changements du commit <id> sans créer de commit
# il faudra le faire nous même.
$ git revert -n <id>
$ git commit -m "Revert"
\end{verbatim}

\paragraph{Attention : } Revert ne fonctionne qu'avec des commits et se termine en créeant un commit !

\newpage
\subsection{Annuler les modifications en cours}

Il y a plusieurs méthodes pour faire ça : checkout et reset :
(l'option hard permet d'appliquer l'annulation sur les fichiers en plus de HEAD)

\begin{verbatim}
$ git reset --hard HEAD
$ git checkout -- .
\end{verbatim}

\subsubsection{Annuler un commit non partagé}
Utilisé sur un commit (et non sur HEAD), cela permet de supprimer le commit de l'arbre.
Si le commit a été déjà partagé, cela ne servira à rien et risque de poser des problèmes, dans ce cas il faudra utiliser revert (mais il apparaîtra toujours dans l'arbre).\\

Vous connaissez maintenant les différences de comportement entre checkout, revert et reset.


