\newpage
\section{Un dépôt en ligne}
L'un des grands avantages d'utiliser git, en plus de pouvoir gérer facilement les versions hors-ligne (ce qui le démarque de \acl{svn}), c'est aussi la possibilité de synchroniser ses branches sur un serveur en ligne rendant ainsi possible la mise en commun des différentes versions\\

Lorsque l'on lie notre dépôt à un dépôt distant, on peut :
\begin{description}
\item[- fetch :]Recevoir les branches en ligne (et les commits de différence)
\item[- push :] Créer/Fusionner (commit de fusion interdit) une de nos branches, dans le serveur
\end{description}

Commençons par lier notre dépôt à un dépôt en ligne.

\subsection{Lier un dépôt local à un dépôt serveur}
\subsubsection{Par ligne de commande}
\begin{verbatim}
$ git remote add <nom identifiant du dépôt> <adresse du dépôt>

# Par défaut on appelle le premier identifiant "origin"

# Exemple pour github, par ssh :
$ git remote add origin git@github.com:nomUtilisateur/nomDepot.git

# Exemple pour github, par https :
$ git remote add origin https://github.com/nomUtilisateur/nomDepot.git
\end{verbatim}

\subsubsection{Par Tortoise git}
Clic droit $\rightarrow$ Tortoise git $\rightarrow$ Settings $\rightarrow$ Git $\rightarrow$ Remote\\

Ici, on renseigne les champs "Remote" et "URL", avant d'appuyer "Add New/Save" puis "Ok"\\

Par exemple : "origin" et "git@github.com:nomUtilisateur/nomDepot.git" pour github

\newpage
\subsection{Créer un clone d'un dépôt en ligne}

Créer un clone d'un dépôt permet d'obtenir un dépôt local identique au dépôt en ligne, directement lié avec le dépôt en ligne. (l'identifiant sera "origin").

\subsubsection{Par ligne de commande}
\begin{verbatim}
$ git clone <adresse du dépot> [nom du dossier (optionnel)]

# Exemple pour github, par ssh :
$ git clone git@github.com:nomUtilisateur/nomDepot.git

# un dossier "nomDepot" sera crée

# Exemple pour github, par ssh :
$ git clone git@github.com:nomUtilisateur/nomDepot.git nomDossier

# un dossier "nomDossier" sera crée
\end{verbatim}

\subsubsection{Tortoise git}
Clic droit en dehors d'un quelconque dépôt, puis Clone.
On remplit l'url, et c'est parti !

\subsection{Récupèrer les différences avec le serveur}
Pour récupèrer les modifications qui se sont faites en ligne, il suffit de faire un "fetch"
(git fetch).

Si des commits existent en ligne mais pas en local, ils seront téléchargés.

Exemple :
\begin{verbatim}

          A     B     C
local :   * --- * --- *     
                  HEAD=master

ORIGIN    A     B     C     D
remote :  * --- * --- * --- *
                        HEAD=master
   
<<FETCH ORIGIN>>

          A        B        C        D         
local :   * ------ * ------ * ------ *
                        HEAD=master
                          origin/HEAD=origin/master
\end{verbatim}

Pour mettre à jour master, il suffit de faire un merge de origin/master !\\

Faire les deux opérations à la suite, c'est faire faire un pull (git pull)

\subsection{Fusionner une branche distante}
Pour mettre à jour le serveur, faire un Push, il faut au préalable que notre branche soit déjà fusionnée avec la branche à fusionner.. concrètement :
\begin{verbatim}
A     B     C
* --- * --- *             <-- Déjà à jour.
          master
       origin/master

A     B     C
* --- * --- *             <-- Possible
          master   
  origin/master
  
A     B     C
* --- * --- *             <-- Pas possible (inutile)
          origin/master   
    master
           
A     B     C     D
* --- * --- *             <-- Pas Possible (non fusionnée : conflit possible)
      |   master  
      ----------- *     
            origin/master
\end{verbatim}

\subsection{Astuce : Le protocole habituel}
Habituellement, lorsque l'on veut envoyer notre version en ligne, on fait, ces opérations dans l'ordre :

\begin{enumerate}
\item ADD : On indexe les modifications et on met en cache les nouveaux fichiers
\item COMMIT : On crée un commit, une version
\item FETCH : On récupère la version en ligne
\item MERGE : On fusionne la version en ligne avec la version locale
\item PUSH : On envoie la nouvelle version locale en ligne\\
\end{enumerate}

Puisque un commit peut faire un "add" et que le fetch et merge se font avec la commande "pull", on peut résumer le protocole à :\\

\begin{enumerate}
\item COMMIT EN INDEXANT (commit -a / les cases de tortoise)
\item PULL (FETCH + MERGE)
\item PUSH
\end{enumerate}
\newpage
\subsection*{Lignes de commande}
\begin{verbatim}
# récupèrer pour toutes les remotes
$ git fetch 

# récupèrer que pour origin
$ git fetch origin

# récupèrer pour toutes les remotes et merger toutes les branches
$ git pull

# récupèrer que pour origin et merger toutes les branches
$ git pull origin

# récupèrer que pour origin et ne merger que master
$ git pull origin master

# tout envoyer sur toutes les remotes
$ git push

# tout envoyer sur origin
$ git push origin

# envoyer master sur origin
$ git push origin master

# envoyer master en tant que "maBranche" sur origin
$ git push origin master:maBranche
\end{verbatim}

\subsection{Supprimer une branche distante}
\subsubsection{Par ligne de commande}
On va utiliser l'option --delete de push :
\begin{verbatim}
# détruit et la branche locale origin/maBranche et la branche maBranche d'origin
$ git push origin --delete maBranche
\end{verbatim}
\subsubsection{Par tortoise git}
Dans le log, clic droit sur une remote/branche et supprimer : une popup apparaîtra pour savoir s'il faut ou non supprimer ET la branche locale (remote/branche) ET la branche "branche" sur le dépôt en ligne.