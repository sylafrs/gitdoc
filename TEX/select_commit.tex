\newpage
\section{Sélectionner un commit à partir d'un autre}
\subsection{Identifier un commit précis}

Comme je vous l'avais dit, il y a plusieurs moyens de se déplacer : \\

Par exemple :\\
\begin{itemize}
\item En écrivant le SHA-1 d'un commit
\item En écrivant le nom d'une branche
\item En écrivant le nom d'un tag\\
\end{itemize}

Mais ce n'est pas tout !\\

On peut aussi :
\begin{itemize}
\item Se déplacer dans un commit à partir d'un autre commit
\item Se déplacer dans un commit en fonction de l'historique (n-ième opération, ou temps)
\end{itemize}

\subsubsection{Se déplacer à partir d'un commit}

Nous avons là deux symboles : \textasciicircum et \textasciitilde\\
HEAD\textasciicircum1 et HEAD\textasciitilde1 signifient la même chose : le commit précédent.\\

Avec un chiffre supérieur à 1, la différence se voit :
\begin{itemize}
\item HEAD\textasciicircum2 signifie : aller au second parent de HEAD (commit de merge)
\item HEAD\textasciitilde2 signifie : aller au (premier) parent du (premier) parent de HEAD\\
\end{itemize}

Ainsi, master\textasciitilde2 et master\textasciicircum\textasciicircum sont équivalents.\\
On peut aussi bien écrire : master\textasciicircum\textasciicircum\textasciitilde3\textasciicircum2 (ce qui revient à master\textasciitilde5\textasciicircum2, pour le coup)\\

\paragraph{Remarque :} Ecrire master\textasciicircum3 serait voué à l'echec.

\newpage
\subsubsection{Se déplacer à partir de l'historique}

Afin de connaître l'historique des références d'une branche ou de HEAD, on peut utiliser : git reflog.\\

Exemple : 
\begin{verbatim}
$ git reflog
734713b... HEAD@{0}: commit: fixed refs handling, added gc auto, updated
d921970... HEAD@{1}: merge phedders/rdocs: Merge made by recursive.
1c002dd... HEAD@{2}: commit: added some blame and merge stuff
1c36188... HEAD@{3}: rebase -i (squash): updating HEAD
95df984... HEAD@{4}: commit: # This is a combination of two commits.
1c36188... HEAD@{5}: rebase -i (squash): updating HEAD
7e05da5... HEAD@{6}: rebase -i (pick): updating HEAD

# git show montre le commit pointé
$ git show HEAD@{3}

# Si vous tapez trop loin, ça ne marchera pas.
\end{verbatim}

Comme vous pouvez le voir : on peut utiliser le symbole @ pour connaître le commit (le sha1 à côté) que pointait la référence au moment donné.

En plus d'utiliser un chiffre pour récupèrer la n-ième opération, on peut aussi bien utiliser une date :

\begin{verbatim}
$ git show HEAD@{yesterday}
$ git show HEAD@{2.months.ago}
$ git show master@{1 month 2 weeks 3 days 1 hour 1 second ago}
$ git show mybirth@{1992-09-29 23:00:00}

# Si vous tapez trop loin, ça ne marchera pas.
\end{verbatim}

\newpage
\subsection{Identifier un ensemble de commits}
\subsubsection*{Les deux points : ..}
Les deux points représentent la portée entre deux commits, par exemple :
branche1..branche2 représente tous les commits qui sont référencés dans branche2 mais pas dans branche1

Exemple :
\begin{verbatim}
A     B     C     D     E     F     G
* --- * --- * --- * --------- *
            |               master
            ----------- * --------- *
                                   test
                                   
$ git log master..test
G
E

$ git log test..master
F
D
\end{verbatim}

\subsubsection*{--not et \textasciicircum}
Certes le symbole .. est utile, mais parfois, on a besoin d'utiliser plus de deux branches.\\
Pour cela on peut utiliser le --not (ou \textasciicircum).

Exemple :
\begin{verbatim}
# Equivalents :
$ git log master..test
$ git log ^master test
$ git log test --not master

# Equivalents :
$ git log master test --not origin/master
$ git log master test ^origin/master

\end{verbatim}

\newpage
\subsubsection*{Les trois points : ...}

Les trois points représentent tous les commits entre deux références sauf ceux qui sont accessible à partir des deux références.\\

Exemple :
\begin{verbatim}
A     B     C     D     E     F     G
* --- * --- * --- * --------- *
            |               master
            ----------- * --------- *
                                   test
                                   
$ git log master...test
D
E
F
G

$ git log test...master
D
E
F
G

$ git log --left-right test...master
> D
< E
> F
< G
\end{verbatim}

\subsubsection{La portée chez Tortoise Git}

En maintenant Ctrl appuyé, dans le log, il est possible de selectionner plusieurs commits.\\
Dans le cas où on en selectionne deux, tortoise git nous propose un sous-log des commits en fonction de .. ou de ... !