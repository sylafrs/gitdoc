\newpage
\section{Ajoutez un dépôt.. ..dans votre dépôt}

%Pour faire cela, nous avons deux méthodes :
%\begin{description}
%\item[- git submodule : ] Va créer un nouveau dépôt git, dans un dossier donné, mais stocke sont propre .git dans le .git du dépôt parent (dans .git/modules). Les différents sous-dépôts seront référencés dans un fichier .gitmodules situé à la racine du dépôt parent. Pour manipuler ce sous dépôt il suffit de s'y rendre et d'utiliser les commandes git (/Tortoise Git) habituelles\\
%\item[- git subtree : ] Méthode la plus élégante et rétro-compatible, mais pas vraiment user-friendly. Pour manipuler tout ça, il faudra passer par de nouvelles commandes.. ..qui ne sont pas intégrées à Tortoise Git. C'est pour cela que je n'en parlerai pas dans ce tutoriel.\\
%\end{description}
%
%Pour en savoir plus sur git subtree et les différences entre ces méthodes, je vous conseille d'aller ici :
%\begin{itemize}
%\item Le man de git-subtree
%\item Alternatives To Git Submodule: Git Subtree 
%\end{itemize}

Pour cela nous allons créer un sous-dépôt, un 'submodule'.\\
Un sous-dépôt est un dépôt dont le .git se trouve dans le .git du parent dans le dossier modules, pour être précis.\\
La racine du sous-dépôt doit être l'un des dossiers du dépôt parent. Les différents sous-dépôts étant référencés dans un fichier .gitmodules se situant à la racine du dossier parent.\\

Voici un schéma de l'arborescence que l'on peut avoir :\\

\dirtree{%
 .1 /.
 .2 CheminVers/.
 .3 MonProjetGit/\\.
 .4 UnDossier1/.
 .4 UnFichier2.
 .4 UnAutreFichier3\\.
 .4 UnAutreDossier4/.
 .5 UnFichier5.
 .5 MonSousDepotGit/.
 .6 .git.
 .6 UnFichier6\\.
 .4 .gitmodules\\.
 .4 .git/.
 .5 modules.
 .6 UnAutreDossier4/.
 .7 MonSousDepotGit/.
 .8 (équivalent du .git du sous dépot)\\.
 .5 (d'autres choses)\\.
}

Ici, Le dossier /CheminVers/MonProjetGit/UnAutreDossier4/MonSousDepotGit est le sous-dépôt du dépôt /CheminVers/MonProjetGit/\\

Dans le .git du dépôt parent se trouve le ".git" du sous-dépôt dans de multiples sous-dossiers vide, en : \\
/CheminVers/MonProjetGit/.git/modules/UnAutreDossier4/MonSousDepotGit/\\

Dans le sous dépot, le .git est un fichier contenant un chemin relatif vers le dossier "MonSousDepotGit" qui se trouve dans le dépôt parent. Ici, il ressemblera à :
\begin{verbatim}
gitdir: ../../.git/modules/UnAutreDossier4/MonSousDepotGit/
\end{verbatim}
\newpage
Dans la racine du dépôt parent se situe un fichier .gitmodules qui liste les sous-dépôts et leur remotes. Ici, il ressemblera à :
\begin{verbatimtab}[4]
[submodule "UnAutreDossier4/MonSousDepotGit"]
	path = UnAutreDossier4/MonSousDepotGit
	url = http://monserveur.fr/MonSousDepot.git
\end{verbatimtab}


Une fois crée il est possible de le manipuler comme un dépôt normal en s'y positionnant (cd/clic droit sur le dossier en question) 
